\section{Future Work}
This paper focused on the architectural design and system properties of \codename, which
leverages on careful interplay multiple state-of-the-art scalability solutions to build a
high performance decentralized exchanges on cryptocurrency systems. The open question that is left for future work is
 how to deal with front running. Miner/validator frontrunning is a well-observed
  problem in permissionless blockchain system and is particularly troubesome in auction system
  and exchange market. Specifically in \codename design, validators in each shard are able to profit from
  "front running" the original order takers by pushing their own filling transactions first and resell later
  at a higher price due to change in liquidity. Proposals like \textit{Submarine Send}\cite{submarine}
  utilizes improvised "Commit-Reveal" paradigm where commit (also called "submarine send") is indistinguishable
  from a normal transaction thus indistinguishable from miners' perspective, given sufficient anonymity-set size. It is
  a viable solution for auction bid and order fill. However, it is still not clear how Submarine prevents validators from committing a lot of fake orders and only reveal the one that benefits them to front-run some specific order. In addition, this solution affects usability as it will introduce delay (confirmation blocks between commit phase
  and reveal phase) and users will have to take extra steps to participate in the protocol. 
  Note that on centralized exchanges, front-running is easier to do, and its hard to even detect if the exchange is front running other users.
