\section{今後の課題}
本稿では、\codenameのアーキテクチャー設計とシステム特性に焦点を当てました。\codenameは、複数の最先端のスケーラビリティソリューションを慎重に相互運用し、暗号通貨システムで高性能な分散型取引所を構築します。将来の仕事のために残されている未解決の問題は、フロントランニングに対処する方法です。マイナー/バリデータのフロントランニングは、誰でも参加できるブロックチェーンシステムでよく見られる問題であり、オークションシステムや交換マーケットでは特に面倒です。特に\codename設計では、各シャードのバリデーターは、自身の取引を最初に挿入して元の受注者を「フロントランニング」し、流動性の変化により後で高額で再販することで利益を得ることができます。\textit{Submarine Send}\cite{submarine}のようなプロポーザルは、十分な匿名性の設定が与えられているため、コミット(submarine sendとも呼ばれる)が通常の取引とマイナーの視点からは区別できない「Commit-Reveal」パラダイムを利用しています。オークションの入札および注文の記入には実行可能なソリューションです。しかし、Submarineがバリデータを使って多くの偽の注文をしないようにする方法は明らかではありません。さらに、このソリューションは、遅延(コミットフェーズと公開フェーズの間の確認ブロック)を発生させるため、ユーザビリティに影響し、ユーザはプロトコルに参加するために追加の手順を踏む必要があります。中央集権的取引所では、フロントランニングは容易に行うことができ、取引所が他のユーザに対してフロントランニングを行っているかを検出することさえ困難です。
